\documentclass[]{article}
\usepackage{lmodern}
\usepackage{amssymb,amsmath}
\usepackage{ifxetex,ifluatex}
\usepackage{fixltx2e} % provides \textsubscript
\ifnum 0\ifxetex 1\fi\ifluatex 1\fi=0 % if pdftex
  \usepackage[T1]{fontenc}
  \usepackage[utf8]{inputenc}
\else % if luatex or xelatex
  \ifxetex
    \usepackage{mathspec}
  \else
    \usepackage{fontspec}
  \fi
  \defaultfontfeatures{Ligatures=TeX,Scale=MatchLowercase}
  \newcommand{\euro}{€}
\fi
% use upquote if available, for straight quotes in verbatim environments
\IfFileExists{upquote.sty}{\usepackage{upquote}}{}
% use microtype if available
\IfFileExists{microtype.sty}{%
\usepackage{microtype}
\UseMicrotypeSet[protrusion]{basicmath} % disable protrusion for tt fonts
}{}
\usepackage[margin=1in]{geometry}
\usepackage{hyperref}
\PassOptionsToPackage{usenames,dvipsnames}{color} % color is loaded by hyperref
\hypersetup{unicode=true,
            pdftitle={project.R},
            pdfauthor={cmay},
            pdfborder={0 0 0},
            breaklinks=true}
\urlstyle{same}  % don't use monospace font for urls
\usepackage{color}
\usepackage{fancyvrb}
\newcommand{\VerbBar}{|}
\newcommand{\VERB}{\Verb[commandchars=\\\{\}]}
\DefineVerbatimEnvironment{Highlighting}{Verbatim}{commandchars=\\\{\}}
% Add ',fontsize=\small' for more characters per line
\usepackage{framed}
\definecolor{shadecolor}{RGB}{248,248,248}
\newenvironment{Shaded}{\begin{snugshade}}{\end{snugshade}}
\newcommand{\KeywordTok}[1]{\textcolor[rgb]{0.13,0.29,0.53}{\textbf{{#1}}}}
\newcommand{\DataTypeTok}[1]{\textcolor[rgb]{0.13,0.29,0.53}{{#1}}}
\newcommand{\DecValTok}[1]{\textcolor[rgb]{0.00,0.00,0.81}{{#1}}}
\newcommand{\BaseNTok}[1]{\textcolor[rgb]{0.00,0.00,0.81}{{#1}}}
\newcommand{\FloatTok}[1]{\textcolor[rgb]{0.00,0.00,0.81}{{#1}}}
\newcommand{\ConstantTok}[1]{\textcolor[rgb]{0.00,0.00,0.00}{{#1}}}
\newcommand{\CharTok}[1]{\textcolor[rgb]{0.31,0.60,0.02}{{#1}}}
\newcommand{\SpecialCharTok}[1]{\textcolor[rgb]{0.00,0.00,0.00}{{#1}}}
\newcommand{\StringTok}[1]{\textcolor[rgb]{0.31,0.60,0.02}{{#1}}}
\newcommand{\VerbatimStringTok}[1]{\textcolor[rgb]{0.31,0.60,0.02}{{#1}}}
\newcommand{\SpecialStringTok}[1]{\textcolor[rgb]{0.31,0.60,0.02}{{#1}}}
\newcommand{\ImportTok}[1]{{#1}}
\newcommand{\CommentTok}[1]{\textcolor[rgb]{0.56,0.35,0.01}{\textit{{#1}}}}
\newcommand{\DocumentationTok}[1]{\textcolor[rgb]{0.56,0.35,0.01}{\textbf{\textit{{#1}}}}}
\newcommand{\AnnotationTok}[1]{\textcolor[rgb]{0.56,0.35,0.01}{\textbf{\textit{{#1}}}}}
\newcommand{\CommentVarTok}[1]{\textcolor[rgb]{0.56,0.35,0.01}{\textbf{\textit{{#1}}}}}
\newcommand{\OtherTok}[1]{\textcolor[rgb]{0.56,0.35,0.01}{{#1}}}
\newcommand{\FunctionTok}[1]{\textcolor[rgb]{0.00,0.00,0.00}{{#1}}}
\newcommand{\VariableTok}[1]{\textcolor[rgb]{0.00,0.00,0.00}{{#1}}}
\newcommand{\ControlFlowTok}[1]{\textcolor[rgb]{0.13,0.29,0.53}{\textbf{{#1}}}}
\newcommand{\OperatorTok}[1]{\textcolor[rgb]{0.81,0.36,0.00}{\textbf{{#1}}}}
\newcommand{\BuiltInTok}[1]{{#1}}
\newcommand{\ExtensionTok}[1]{{#1}}
\newcommand{\PreprocessorTok}[1]{\textcolor[rgb]{0.56,0.35,0.01}{\textit{{#1}}}}
\newcommand{\AttributeTok}[1]{\textcolor[rgb]{0.77,0.63,0.00}{{#1}}}
\newcommand{\RegionMarkerTok}[1]{{#1}}
\newcommand{\InformationTok}[1]{\textcolor[rgb]{0.56,0.35,0.01}{\textbf{\textit{{#1}}}}}
\newcommand{\WarningTok}[1]{\textcolor[rgb]{0.56,0.35,0.01}{\textbf{\textit{{#1}}}}}
\newcommand{\AlertTok}[1]{\textcolor[rgb]{0.94,0.16,0.16}{{#1}}}
\newcommand{\ErrorTok}[1]{\textcolor[rgb]{0.64,0.00,0.00}{\textbf{{#1}}}}
\newcommand{\NormalTok}[1]{{#1}}
\usepackage{graphicx,grffile}
\makeatletter
\def\maxwidth{\ifdim\Gin@nat@width>\linewidth\linewidth\else\Gin@nat@width\fi}
\def\maxheight{\ifdim\Gin@nat@height>\textheight\textheight\else\Gin@nat@height\fi}
\makeatother
% Scale images if necessary, so that they will not overflow the page
% margins by default, and it is still possible to overwrite the defaults
% using explicit options in \includegraphics[width, height, ...]{}
\setkeys{Gin}{width=\maxwidth,height=\maxheight,keepaspectratio}
\setlength{\parindent}{0pt}
\setlength{\parskip}{6pt plus 2pt minus 1pt}
\setlength{\emergencystretch}{3em}  % prevent overfull lines
\providecommand{\tightlist}{%
  \setlength{\itemsep}{0pt}\setlength{\parskip}{0pt}}
\setcounter{secnumdepth}{0}

%%% Use protect on footnotes to avoid problems with footnotes in titles
\let\rmarkdownfootnote\footnote%
\def\footnote{\protect\rmarkdownfootnote}

%%% Change title format to be more compact
\usepackage{titling}

% Create subtitle command for use in maketitle
\newcommand{\subtitle}[1]{
  \posttitle{
    \begin{center}\large#1\end{center}
    }
}

\setlength{\droptitle}{-2em}
  \title{project.R}
  \pretitle{\vspace{\droptitle}\centering\huge}
  \posttitle{\par}
  \author{cmay}
  \preauthor{\centering\large\emph}
  \postauthor{\par}
  \predate{\centering\large\emph}
  \postdate{\par}
  \date{Wed Mar 14 13:42:00 2018}


% Redefines (sub)paragraphs to behave more like sections
\ifx\paragraph\undefined\else
\let\oldparagraph\paragraph
\renewcommand{\paragraph}[1]{\oldparagraph{#1}\mbox{}}
\fi
\ifx\subparagraph\undefined\else
\let\oldsubparagraph\subparagraph
\renewcommand{\subparagraph}[1]{\oldsubparagraph{#1}\mbox{}}
\fi


\begin{document}
\maketitle

\begin{Shaded}
\begin{Highlighting}[]
\NormalTok{#### END OF COURSE PROJECT ####}
\KeywordTok{library}\NormalTok{(ggplot2)}
\end{Highlighting}
\end{Shaded}

\begin{verbatim}
## Warning: package 'ggplot2' was built under R version 3.3.2
\end{verbatim}

\begin{Shaded}
\begin{Highlighting}[]
\KeywordTok{library}\NormalTok{(caret)}
\end{Highlighting}
\end{Shaded}

\begin{verbatim}
## Warning: package 'caret' was built under R version 3.3.2
\end{verbatim}

\begin{verbatim}
## Loading required package: lattice
\end{verbatim}

\begin{verbatim}
## Warning in as.POSIXlt.POSIXct(Sys.time()): unknown timezone 'zone/tz/2018c.
## 1.0/zoneinfo/America/New_York'
\end{verbatim}

\begin{Shaded}
\begin{Highlighting}[]
\KeywordTok{library}\NormalTok{(kernlab)}
\end{Highlighting}
\end{Shaded}

\begin{verbatim}
## 
## Attaching package: 'kernlab'
\end{verbatim}

\begin{verbatim}
## The following object is masked from 'package:ggplot2':
## 
##     alpha
\end{verbatim}

\begin{Shaded}
\begin{Highlighting}[]
\KeywordTok{library}\NormalTok{(gridExtra)}
\end{Highlighting}
\end{Shaded}

\begin{verbatim}
## Warning: package 'gridExtra' was built under R version 3.3.2
\end{verbatim}

\begin{Shaded}
\begin{Highlighting}[]
\NormalTok{## Read in and prepare data}
\NormalTok{pmltrain =}\StringTok{ }\KeywordTok{read.csv}\NormalTok{(}\StringTok{"/Users/cmay/Documents/Training/pml-training.csv"}\NormalTok{, }\DataTypeTok{stringsAsFactors=}\OtherTok{FALSE}\NormalTok{)}
\NormalTok{pmltest  =}\StringTok{ }\KeywordTok{read.csv}\NormalTok{(}\StringTok{"/Users/cmay/Documents/Training/pml-testing.csv"}\NormalTok{,  }\DataTypeTok{stringsAsFactors=}\OtherTok{FALSE}\NormalTok{)}
  \KeywordTok{names}\NormalTok{(pmltrain)[}\KeywordTok{names}\NormalTok{(pmltrain)==}\StringTok{'X'}\NormalTok{] <-}\StringTok{ 'id'}
  \KeywordTok{names}\NormalTok{(pmltest )[}\KeywordTok{names}\NormalTok{(pmltest) ==}\StringTok{'X'}\NormalTok{] <-}\StringTok{ 'id'}

  \KeywordTok{table}\NormalTok{(pmltrain$classe, }\DataTypeTok{exclude=}\OtherTok{FALSE}\NormalTok{)}
\end{Highlighting}
\end{Shaded}

\begin{verbatim}
## 
##    A    B    C    D    E 
## 5580 3797 3422 3216 3607
\end{verbatim}

\begin{Shaded}
\begin{Highlighting}[]
  \KeywordTok{table}\NormalTok{(pmltest$classe)}
\end{Highlighting}
\end{Shaded}

\begin{verbatim}
## < table of extent 0 >
\end{verbatim}

\begin{Shaded}
\begin{Highlighting}[]
  \KeywordTok{nrow}\NormalTok{(pmltrain[}\KeywordTok{is.na}\NormalTok{(pmltrain$classe),])}
\end{Highlighting}
\end{Shaded}

\begin{verbatim}
## [1] 0
\end{verbatim}

\begin{Shaded}
\begin{Highlighting}[]
  \KeywordTok{nrow}\NormalTok{(pmltrain[pmltrain$classe ==}\StringTok{ " "}\NormalTok{,])}
\end{Highlighting}
\end{Shaded}

\begin{verbatim}
## [1] 0
\end{verbatim}

\begin{Shaded}
\begin{Highlighting}[]
\NormalTok{## Get the number of NA records in a dataframes of columns}
\NormalTok{countnas <-}\StringTok{ }\NormalTok{function(z) \{}
  \NormalTok{na_count <-}\KeywordTok{sapply}\NormalTok{(z, function(y) }\KeywordTok{sum}\NormalTok{(}\KeywordTok{length}\NormalTok{(}\KeywordTok{which}\NormalTok{(}\KeywordTok{is.na}\NormalTok{(y) |}\StringTok{ }\NormalTok{y ==}\StringTok{ ""}\NormalTok{))))}
  \NormalTok{na_count <-}\StringTok{ }\KeywordTok{data.frame}\NormalTok{(na_count)}
  \NormalTok{na_count$pct <-}\StringTok{ }\KeywordTok{round}\NormalTok{(na_count$na_count  /}\StringTok{ }\KeywordTok{nrow}\NormalTok{(z),}\DecValTok{4}\NormalTok{)}
  \KeywordTok{return}\NormalTok{(na_count)}
\NormalTok{\}}

\NormalTok{## Writeout table for report}
\CommentTok{#countnas(pmltrain)}
\CommentTok{#outtrain <- countnas(pmltrain)}
\CommentTok{#write.csv(cbind(names(pmltrain),outtrain),file="/Users/cmay/Documents/Training/outtrain.csv", quote=FALSE, row.names=FALSE)}


\NormalTok{## Remove vars with at least one NA, most have 19K NA's}
\NormalTok{## Set sample seed here.  This will be run 3 times by changing the seed to create different samples}
\NormalTok{## Could also build a function for all of this work if I want to}
  \NormalTok{## 1) seed 41;}
  \NormalTok{## 2) seed 92621;}
  \NormalTok{## 3) seed 52800}
\KeywordTok{set.seed}\NormalTok{(}\DecValTok{52800}\NormalTok{) }
\NormalTok{pmltrain.clean <-}\StringTok{ }\NormalTok{pmltrain[ , }\KeywordTok{apply}\NormalTok{(pmltrain, }\DecValTok{2}\NormalTok{, function(x) \{!}\KeywordTok{any}\NormalTok{(}\KeywordTok{is.na}\NormalTok{(x)) &}\StringTok{ }\NormalTok{!}\KeywordTok{any}\NormalTok{( x ==}\StringTok{ ""}\NormalTok{) \} )]}
\NormalTok{pmltest.clean  <-}\StringTok{ }\NormalTok{pmltest[ ,  }\KeywordTok{apply}\NormalTok{(pmltest, }\DecValTok{2}\NormalTok{, function(x) \{!}\KeywordTok{any}\NormalTok{(}\KeywordTok{is.na}\NormalTok{(x)) &}\StringTok{ }\NormalTok{!}\KeywordTok{any}\NormalTok{( x ==}\StringTok{ ""}\NormalTok{) \} )]}

\NormalTok{## Remove character, id and timestamp variables from final test set}
\NormalTok{pmltest.clean  <-}\StringTok{ }\NormalTok{pmltest.clean[,-}\KeywordTok{c}\NormalTok{(}\DecValTok{1}\NormalTok{:}\DecValTok{7}\NormalTok{)]}
\CommentTok{#str(pmltest.clean)}

\NormalTok{## Build cross-validation training and test sets}
\NormalTok{intrain  =}\StringTok{ }\KeywordTok{createDataPartition}\NormalTok{(pmltrain.clean$classe, }\DataTypeTok{p=}\DecValTok{3}\NormalTok{/}\DecValTok{4}\NormalTok{)[[}\DecValTok{1}\NormalTok{]]}
\NormalTok{training =}\StringTok{ }\NormalTok{pmltrain.clean[ intrain, ]}
\NormalTok{testing  =}\StringTok{ }\NormalTok{pmltrain.clean[-intrain, ]}
\CommentTok{#str(training); str(testing)}


\NormalTok{## Remove indicator, character and time variables from training data}
\NormalTok{training <-}\StringTok{ }\NormalTok{training[,-}\KeywordTok{c}\NormalTok{(}\DecValTok{1}\NormalTok{:}\DecValTok{7}\NormalTok{)]}
\CommentTok{#str(training); summary(training); head(training,4)}
\CommentTok{#nzv <- nearZeroVar(training, saveMetrics=TRUE)}
\CommentTok{#nzv}

\NormalTok{## Prepare 'clean' testing data set for trained models}
\NormalTok{testing.clean <-}\StringTok{ }\NormalTok{testing[ , }\KeywordTok{apply}\NormalTok{(testing, }\DecValTok{2}\NormalTok{, function(x) \{!}\KeywordTok{any}\NormalTok{(}\KeywordTok{is.na}\NormalTok{(x)) &}\StringTok{ }\NormalTok{!}\KeywordTok{any}\NormalTok{( x ==}\StringTok{ ""}\NormalTok{) \} )]}
\NormalTok{testing.clean <-}\StringTok{ }\NormalTok{testing.clean[,-}\KeywordTok{c}\NormalTok{(}\DecValTok{1}\NormalTok{:}\DecValTok{7}\NormalTok{)]}
\CommentTok{#str(testing.clean)}



\NormalTok{##########################################################################################}
\NormalTok{#### Exploratory Data Analysis}
\NormalTok{## Change 'turn.on' value to 1 if I want to view plots again}
\NormalTok{turn.on <-}\StringTok{ }\DecValTok{0}
\NormalTok{if (turn.on ==}\StringTok{ }\DecValTok{1}\NormalTok{) \{}
  \KeywordTok{featurePlot}\NormalTok{(}\DataTypeTok{x=} \NormalTok{training$classe, }\DataTypeTok{y=} \NormalTok{training[,}\KeywordTok{c}\NormalTok{(}\StringTok{'roll_belt'}\NormalTok{, }\StringTok{'pitch_belt'}\NormalTok{, }\StringTok{'yaw_belt'}\NormalTok{)], }\DataTypeTok{plot=}\StringTok{'pairs'}\NormalTok{)}
  \KeywordTok{qplot}\NormalTok{(classe, magnet_arm_x, }\DataTypeTok{data=}\NormalTok{training)}
  \KeywordTok{qplot}\NormalTok{(magnet_arm_x,roll_dumbbell, }\DataTypeTok{col=}\NormalTok{classe,   }\DataTypeTok{data=}\NormalTok{training)}
  \KeywordTok{qplot}\NormalTok{(classe, total_accel_forearm, }\DataTypeTok{fill=}\NormalTok{classe, }\DataTypeTok{data=}\NormalTok{training, }\DataTypeTok{geom=}\KeywordTok{c}\NormalTok{(}\StringTok{"boxplot"}\NormalTok{))}
  \KeywordTok{qplot}\NormalTok{(magnet_forearm_z, }\DataTypeTok{col=}\NormalTok{classe, }\DataTypeTok{data=}\NormalTok{training,}\DataTypeTok{geom=}\StringTok{"density"}\NormalTok{)}
  \KeywordTok{plot}\NormalTok{(training$classe, training$roll_belt)}
  
  \NormalTok{## Build PDF files of different types of eda plots by variable}
  \KeywordTok{pdf}\NormalTok{(}\DataTypeTok{file=}\StringTok{"/Users/cmay/Downloads/densityplots.pdf"}\NormalTok{)}
  \KeywordTok{par}\NormalTok{(}\DataTypeTok{mar=}\KeywordTok{c}\NormalTok{(}\DecValTok{3}\NormalTok{,}\DecValTok{3}\NormalTok{,}\DecValTok{2}\NormalTok{,}\DecValTok{0}\NormalTok{))}
  \NormalTok{for (k in }\DecValTok{1}\NormalTok{:}\KeywordTok{length}\NormalTok{(training)) \{}
    \KeywordTok{print}\NormalTok{(}\KeywordTok{qplot}\NormalTok{(training[,k], }\DataTypeTok{col=}\NormalTok{classe, }\DataTypeTok{data=}\NormalTok{training, }\DataTypeTok{geom=}\StringTok{"density"}\NormalTok{, }\DataTypeTok{xlab=}\KeywordTok{colnames}\NormalTok{(training[k]), }
                \DataTypeTok{main=}\KeywordTok{paste0}\NormalTok{(}\StringTok{"Variable #"}\NormalTok{,k) ) )}
  \NormalTok{\}}
  \KeywordTok{dev.off}\NormalTok{()}
  
  \KeywordTok{pdf}\NormalTok{(}\DataTypeTok{file=}\StringTok{"/Users/cmay/Downloads/boxplots.pdf"}\NormalTok{)}
  \KeywordTok{par}\NormalTok{(}\DataTypeTok{mar=}\KeywordTok{c}\NormalTok{(}\DecValTok{3}\NormalTok{,}\DecValTok{3}\NormalTok{,}\DecValTok{2}\NormalTok{,}\DecValTok{0}\NormalTok{))}
  \NormalTok{for (k in }\DecValTok{1}\NormalTok{:(}\KeywordTok{length}\NormalTok{(training)-}\DecValTok{1}\NormalTok{) ) \{}
    \NormalTok{p1 <-}\StringTok{ }\KeywordTok{qplot}\NormalTok{(classe, training[,k], }\DataTypeTok{fill=}\NormalTok{classe, }\DataTypeTok{data=}\NormalTok{training, }\DataTypeTok{geom=}\KeywordTok{c}\NormalTok{(}\StringTok{"boxplot"}\NormalTok{), }\DataTypeTok{ylab=}\KeywordTok{colnames}\NormalTok{(training[k]),}
                \DataTypeTok{main=}\KeywordTok{paste0}\NormalTok{(}\StringTok{"Boxplot of "}\NormalTok{, }\KeywordTok{colnames}\NormalTok{(training[k]) ) )}
    
    \NormalTok{p2 <-}\StringTok{ }\KeywordTok{qplot}\NormalTok{(classe, training[,k], }\DataTypeTok{fill=}\NormalTok{classe, }\DataTypeTok{data=}\NormalTok{training, }\DataTypeTok{geom=}\KeywordTok{c}\NormalTok{(}\StringTok{"boxplot"}\NormalTok{, }\StringTok{"jitter"}\NormalTok{), }\DataTypeTok{ylab=}\KeywordTok{colnames}\NormalTok{(training[k]),}
                \DataTypeTok{main=}\KeywordTok{paste0}\NormalTok{(}\StringTok{"Boxplot of "}\NormalTok{, }\KeywordTok{colnames}\NormalTok{(training[k]) ) )}
    
    \KeywordTok{grid.arrange}\NormalTok{(p1,p2,}\DataTypeTok{ncol=}\DecValTok{2}\NormalTok{)  }
  \NormalTok{\}}
  \KeywordTok{dev.off}\NormalTok{()}
  
  \KeywordTok{pdf}\NormalTok{(}\DataTypeTok{file=}\StringTok{"/Users/cmay/Downloads/histograms.pdf"}\NormalTok{)}
  \KeywordTok{par}\NormalTok{(}\DataTypeTok{mar=}\KeywordTok{c}\NormalTok{(}\DecValTok{3}\NormalTok{,}\DecValTok{3}\NormalTok{,}\DecValTok{2}\NormalTok{,}\DecValTok{0}\NormalTok{))}
  \NormalTok{for (k in }\DecValTok{1}\NormalTok{:(}\KeywordTok{length}\NormalTok{(training)-}\DecValTok{1}\NormalTok{)) \{}
    \KeywordTok{hist}\NormalTok{(training[,k], }\DataTypeTok{breaks=}\DecValTok{100}\NormalTok{, }\DataTypeTok{main=}\KeywordTok{paste0}\NormalTok{(}\StringTok{"Histogram of "}\NormalTok{,}\KeywordTok{colnames}\NormalTok{(training[k])))}
  \NormalTok{\}}
  \KeywordTok{dev.off}\NormalTok{()}
  
  \KeywordTok{pdf}\NormalTok{(}\DataTypeTok{file=}\StringTok{"/Users/cmay/Downloads/histograms_log.pdf"}\NormalTok{)}
  \KeywordTok{par}\NormalTok{(}\DataTypeTok{mar=}\KeywordTok{c}\NormalTok{(}\DecValTok{3}\NormalTok{,}\DecValTok{3}\NormalTok{,}\DecValTok{2}\NormalTok{,}\DecValTok{0}\NormalTok{))}
  \NormalTok{for (k in }\DecValTok{1}\NormalTok{:(}\KeywordTok{length}\NormalTok{(training)-}\DecValTok{1}\NormalTok{)) \{}
    \KeywordTok{hist}\NormalTok{(}\KeywordTok{log10}\NormalTok{(training[,k]), }\DataTypeTok{main=}\KeywordTok{paste0}\NormalTok{(}\StringTok{"Histogram of "}\NormalTok{,}\KeywordTok{colnames}\NormalTok{(training[k])))}
  \NormalTok{\}}
  \KeywordTok{dev.off}\NormalTok{()}
\NormalTok{\}}


\NormalTok{## Remove predictors that have little to know relationship with outcome variable after eda}
\NormalTok{## May not use this in the final version}
\NormalTok{training2 <-}\StringTok{ }\KeywordTok{subset}\NormalTok{(training, }\DataTypeTok{select=}\NormalTok{-}\KeywordTok{c}\NormalTok{(gyros_belt_x,gyros_belt_y,gyros_belt_z,gyros_arm_x,gyros_arm_y,gyros_arm_z,}
                                        \NormalTok{gyros_dumbbell_x,gyros_dumbbell_y,gyros_dumbbell_z,gyros_forearm_x,gyros_forearm_y,gyros_forearm_z) )}

\NormalTok{testing2 <-}\StringTok{ }\KeywordTok{subset}\NormalTok{(testing.clean, }\DataTypeTok{select=}\NormalTok{-}\KeywordTok{c}\NormalTok{(gyros_belt_x,gyros_belt_y,gyros_belt_z,gyros_arm_x,gyros_arm_y,gyros_arm_z,}
                                            \NormalTok{gyros_dumbbell_x,gyros_dumbbell_y,gyros_dumbbell_z,gyros_forearm_x,gyros_forearm_y,gyros_forearm_z) )}
\CommentTok{#str(training2); str(testing2)}

\NormalTok{##########################################################################################}
\NormalTok{#### Principal Component Analysis ####}
\NormalTok{#### Went through this but did not use PCA in my final report ####}
\CommentTok{#m <- abs(cor(training[,-c(1:6,60)]))}
\CommentTok{#diag(m) <- 0}
\CommentTok{#which(m > 0.90, arr.ind=T)}
\CommentTok{#plot(training$accel_belt_x, training$pitch_belt)}
\CommentTok{#plot(training$accel_belt_z, training$roll_belt)}
\CommentTok{#X <- -0.12*training$total_accel_belt - 0.99*training$roll_belt}
\CommentTok{#Y <- -0.99*training$total_accel_belt + 0.12*training$roll_belt}
\CommentTok{#plot(X,Y)}

\CommentTok{#typeColor <- training$classe}
\CommentTok{#table(typeColor)}
\CommentTok{#typeColor[typeColor =="A"] <- 1}
\CommentTok{#typeColor[typeColor =="B"] <- 2}
\CommentTok{#typeColor[typeColor =="C"] <- 3}
\CommentTok{#typeColor[typeColor =="D"] <- 4}
\CommentTok{#typeColor[typeColor =="E"] <- 5}

\NormalTok{## PCA- full set of variables for}
\CommentTok{#nrow(pmltrain)}
\CommentTok{#prcomp <- prcomp(training)}
\CommentTok{#prcomp$rotation}
\CommentTok{#str(prcomp)}
\CommentTok{#prcomp}

\CommentTok{#plot(prcomp$x[,1], prcomp$x[,2], xlab="PC1",ylab="PC2")}
\CommentTok{#plot(prcomp$x[,1], prcomp$x[,2], col=typeColor, xlab="PC1",ylab="PC2", cex=1)}
\CommentTok{#classPC <- predict(prcomp,training)}
\CommentTok{#plot(classPC[,1], classPC[,2], col=typeColor, xlab="PC1",ylab="PC2")}


\NormalTok{## PCA- reduced set of variables with caret package for testing}
\CommentTok{#smallset <- training2[,c(1:10)]}
\CommentTok{#str(smallset)}
\CommentTok{#preproc <- preProcess(training2, method="pca", thresh=0.90)}
\CommentTok{#str(preproc)}
\CommentTok{#preproc$rotation}
\CommentTok{#plot(preproc$x[,1], preproc$x[,2], col=typeColor, xlab="PC1",ylab="PC2")}


\NormalTok{##########################################################################################}
\NormalTok{#### Build models}
\NormalTok{#### Rpart- classification and regression tree}
\KeywordTok{print}\NormalTok{(}\StringTok{"Start of RPART"}\NormalTok{)}
\end{Highlighting}
\end{Shaded}

\begin{verbatim}
## [1] "Start of RPART"
\end{verbatim}

\begin{Shaded}
\begin{Highlighting}[]
\KeywordTok{Sys.time}\NormalTok{()}
\end{Highlighting}
\end{Shaded}

\begin{verbatim}
## [1] "2018-03-14 17:42:07 GMT"
\end{verbatim}

\begin{Shaded}
\begin{Highlighting}[]
\NormalTok{mod.rpart <-}\StringTok{ }\KeywordTok{train}\NormalTok{(classe ~}\StringTok{ }\NormalTok{., }\DataTypeTok{data=}\NormalTok{training, }\DataTypeTok{method=}\StringTok{"rpart"}\NormalTok{)}
\NormalTok{mod.rpart$finalModel}
\end{Highlighting}
\end{Shaded}

\begin{verbatim}
## n= 14718 
## 
## node), split, n, loss, yval, (yprob)
##       * denotes terminal node
## 
##  1) root 14718 10533 A (0.28 0.19 0.17 0.16 0.18)  
##    2) roll_belt< 130.5 13492  9315 A (0.31 0.21 0.19 0.18 0.11)  
##      4) pitch_forearm< -33.65 1193     9 A (0.99 0.0075 0 0 0) *
##      5) pitch_forearm>=-33.65 12299  9306 A (0.24 0.23 0.21 0.2 0.12)  
##       10) magnet_dumbbell_y< 438.5 10407  7472 A (0.28 0.18 0.24 0.19 0.11)  
##         20) roll_forearm< 123.5 6505  3858 A (0.41 0.18 0.18 0.17 0.061) *
##         21) roll_forearm>=123.5 3902  2610 C (0.074 0.18 0.33 0.23 0.19) *
##       11) magnet_dumbbell_y>=438.5 1892   925 B (0.031 0.51 0.042 0.22 0.19) *
##    3) roll_belt>=130.5 1226     8 E (0.0065 0 0 0 0.99) *
\end{verbatim}

\begin{Shaded}
\begin{Highlighting}[]
\NormalTok{## Get predictions and confusion(accuracy matrix) for rpart}
\NormalTok{pred.rpart <-}\StringTok{ }\KeywordTok{predict}\NormalTok{(mod.rpart, testing)}
\KeywordTok{confusionMatrix}\NormalTok{(testing$classe, pred.rpart)}
\end{Highlighting}
\end{Shaded}

\begin{verbatim}
## Confusion Matrix and Statistics
## 
##           Reference
## Prediction    A    B    C    D    E
##          A 1247   29  113    0    6
##          B  396  326  227    0    0
##          C  391   29  435    0    0
##          D  356  148  300    0    0
##          E  126  127  235    0  413
## 
## Overall Statistics
##                                           
##                Accuracy : 0.4937          
##                  95% CI : (0.4796, 0.5078)
##     No Information Rate : 0.5131          
##     P-Value [Acc > NIR] : 0.9968          
##                                           
##                   Kappa : 0.3388          
##  Mcnemar's Test P-Value : NA              
## 
## Statistics by Class:
## 
##                      Class: A Class: B Class: C Class: D Class: E
## Sensitivity            0.4956  0.49469   0.3321       NA  0.98568
## Specificity            0.9380  0.85324   0.8831   0.8361  0.89119
## Pos Pred Value         0.8939  0.34352   0.5088       NA  0.45838
## Neg Pred Value         0.6384  0.91580   0.7839       NA  0.99850
## Prevalence             0.5131  0.13438   0.2671   0.0000  0.08544
## Detection Rate         0.2543  0.06648   0.0887   0.0000  0.08422
## Detection Prevalence   0.2845  0.19352   0.1743   0.1639  0.18373
## Balanced Accuracy      0.7168  0.67396   0.6076       NA  0.93844
\end{verbatim}

\begin{Shaded}
\begin{Highlighting}[]
\NormalTok{## See what predictions look like on "real" test set}
\NormalTok{pred.rpart2 <-}\StringTok{ }\KeywordTok{predict}\NormalTok{(mod.rpart, pmltest.clean)}
\NormalTok{pred.rpart2}
\end{Highlighting}
\end{Shaded}

\begin{verbatim}
##  [1] C A C A A C C A A A C C C A C A A A A C
## Levels: A B C D E
\end{verbatim}

\begin{Shaded}
\begin{Highlighting}[]
\NormalTok{## Save a plot of the tree}
\NormalTok{if (turn.on ==}\StringTok{ }\DecValTok{1}\NormalTok{) \{}
  \KeywordTok{pdf}\NormalTok{(}\DataTypeTok{file=}\StringTok{"/Users/cmay/Downloads/rpart_tree92621.pdf"}\NormalTok{)}
  \KeywordTok{par}\NormalTok{(}\DataTypeTok{mar=}\KeywordTok{c}\NormalTok{(}\DecValTok{1}\NormalTok{,}\DecValTok{4}\NormalTok{,}\DecValTok{1}\NormalTok{,}\DecValTok{4}\NormalTok{))}
  \KeywordTok{plot}\NormalTok{(mod.rpart$finalModel, }\DataTypeTok{uniform=}\OtherTok{TRUE}\NormalTok{, }\DataTypeTok{main=}\StringTok{"Classification Tree"}\NormalTok{, }\DataTypeTok{cex.main=}\FloatTok{0.75}\NormalTok{)}
  \KeywordTok{text}\NormalTok{(mod.rpart$finalModel, }\DataTypeTok{use.n=}\OtherTok{TRUE}\NormalTok{, }\DataTypeTok{all=}\OtherTok{TRUE}\NormalTok{, }\DataTypeTok{cex=}\FloatTok{0.7}\NormalTok{,}\DataTypeTok{pos=}\DecValTok{1}\NormalTok{)}
  \KeywordTok{dev.off}\NormalTok{()}
\NormalTok{\}}

\KeywordTok{print}\NormalTok{(}\StringTok{"Start of RANDOM FORESTS- reduced set of variables, testing for time "}\NormalTok{)}
\end{Highlighting}
\end{Shaded}

\begin{verbatim}
## [1] "Start of RANDOM FORESTS- reduced set of variables, testing for time "
\end{verbatim}

\begin{Shaded}
\begin{Highlighting}[]
\KeywordTok{Sys.time}\NormalTok{()}
\end{Highlighting}
\end{Shaded}

\begin{verbatim}
## [1] "2018-03-14 17:42:25 GMT"
\end{verbatim}

\begin{Shaded}
\begin{Highlighting}[]
\NormalTok{#### RF- Random Forest}
\CommentTok{#mod.rf <- train(classe ~ roll_belt + yaw_belt + total_accel_belt + accel_belt_y + accel_belt_z + magnet_belt_y + magnet_belt_z +}
\CommentTok{#                              roll_arm + pitch_arm + total_accel_arm + accel_arm_x + accel_arm_y + accel_arm_z + magnet_arm_x +}
\CommentTok{#                              magnet_arm_y + magnet_arm_z + roll_dumbbell + pitch_dumbbell, data=training2, method="rf")}
\NormalTok{mod.rf <-}\StringTok{ }\KeywordTok{train}\NormalTok{(classe ~}\StringTok{ }\NormalTok{., }\DataTypeTok{data=}\NormalTok{training2, }\DataTypeTok{method=}\StringTok{"rf"}\NormalTok{)}
\KeywordTok{summary}\NormalTok{(mod.rf)}
\end{Highlighting}
\end{Shaded}

\begin{verbatim}
##                 Length Class      Mode     
## call                4  -none-     call     
## type                1  -none-     character
## predicted       14718  factor     numeric  
## err.rate         3000  -none-     numeric  
## confusion          30  -none-     numeric  
## votes           73590  matrix     numeric  
## oob.times       14718  -none-     numeric  
## classes             5  -none-     character
## importance         40  -none-     numeric  
## importanceSD        0  -none-     NULL     
## localImportance     0  -none-     NULL     
## proximity           0  -none-     NULL     
## ntree               1  -none-     numeric  
## mtry                1  -none-     numeric  
## forest             14  -none-     list     
## y               14718  factor     numeric  
## test                0  -none-     NULL     
## inbag               0  -none-     NULL     
## xNames             40  -none-     character
## problemType         1  -none-     character
## tuneValue           1  data.frame list     
## obsLevels           5  -none-     character
## param               0  -none-     list
\end{verbatim}

\begin{Shaded}
\begin{Highlighting}[]
\NormalTok{mod.rf$finalModel}
\end{Highlighting}
\end{Shaded}

\begin{verbatim}
## 
## Call:
##  randomForest(x = x, y = y, mtry = param$mtry) 
##                Type of random forest: classification
##                      Number of trees: 500
## No. of variables tried at each split: 2
## 
##         OOB estimate of  error rate: 0.79%
## Confusion matrix:
##      A    B    C    D    E class.error
## A 4178    5    2    0    0 0.001672640
## B   22 2819    7    0    0 0.010182584
## C    0   25 2538    4    0 0.011297234
## D    0    0   37 2372    3 0.016583748
## E    0    1    4    6 2695 0.004065041
\end{verbatim}

\begin{Shaded}
\begin{Highlighting}[]
\NormalTok{## Get predictions and confusion(accuracy matrix) for rf}
\NormalTok{pred.rf <-}\StringTok{ }\KeywordTok{predict}\NormalTok{(mod.rf, testing2)}
\KeywordTok{confusionMatrix}\NormalTok{(testing2$classe, pred.rf)}
\end{Highlighting}
\end{Shaded}

\begin{verbatim}
## Confusion Matrix and Statistics
## 
##           Reference
## Prediction    A    B    C    D    E
##          A 1392    1    2    0    0
##          B   10  936    3    0    0
##          C    0   14  838    3    0
##          D    0    0   13  791    0
##          E    0    0    0    9  892
## 
## Overall Statistics
##                                           
##                Accuracy : 0.9888          
##                  95% CI : (0.9854, 0.9915)
##     No Information Rate : 0.2859          
##     P-Value [Acc > NIR] : < 2.2e-16       
##                                           
##                   Kappa : 0.9858          
##  Mcnemar's Test P-Value : NA              
## 
## Statistics by Class:
## 
##                      Class: A Class: B Class: C Class: D Class: E
## Sensitivity            0.9929   0.9842   0.9790   0.9851   1.0000
## Specificity            0.9991   0.9967   0.9958   0.9968   0.9978
## Pos Pred Value         0.9978   0.9863   0.9801   0.9838   0.9900
## Neg Pred Value         0.9972   0.9962   0.9956   0.9971   1.0000
## Prevalence             0.2859   0.1939   0.1746   0.1637   0.1819
## Detection Rate         0.2838   0.1909   0.1709   0.1613   0.1819
## Detection Prevalence   0.2845   0.1935   0.1743   0.1639   0.1837
## Balanced Accuracy      0.9960   0.9905   0.9874   0.9909   0.9989
\end{verbatim}

\begin{Shaded}
\begin{Highlighting}[]
\NormalTok{## See what predictions look like on "real" test set}
\NormalTok{pred.rf2 <-}\StringTok{ }\KeywordTok{predict}\NormalTok{(mod.rf, pmltest.clean)}
\NormalTok{pred.rf2}
\end{Highlighting}
\end{Shaded}

\begin{verbatim}
##  [1] B A B A A E D B A A B C B A E E A B B B
## Levels: A B C D E
\end{verbatim}

\begin{Shaded}
\begin{Highlighting}[]
\KeywordTok{print}\NormalTok{(}\StringTok{"Start of GBM"}\NormalTok{)}
\end{Highlighting}
\end{Shaded}

\begin{verbatim}
## [1] "Start of GBM"
\end{verbatim}

\begin{Shaded}
\begin{Highlighting}[]
\KeywordTok{Sys.time}\NormalTok{()}
\end{Highlighting}
\end{Shaded}

\begin{verbatim}
## [1] "2018-03-14 18:43:32 GMT"
\end{verbatim}

\begin{Shaded}
\begin{Highlighting}[]
\NormalTok{#### GBM- boosting with trees}
\CommentTok{#mod.gbm <- train(classe ~ roll_belt + yaw_belt + total_accel_belt + accel_belt_y + accel_belt_z + magnet_belt_y + magnet_belt_z +}
\CommentTok{#                              roll_arm + pitch_arm + total_accel_arm + accel_arm_x + accel_arm_y + accel_arm_z + magnet_arm_x +}
\CommentTok{#                              magnet_arm_y + magnet_arm_z + roll_dumbbell + pitch_dumbbell, data=training2, method="gbm", verbose=FALSE)}
\NormalTok{mod.gbm <-}\StringTok{ }\KeywordTok{train}\NormalTok{(classe ~}\StringTok{ }\NormalTok{., }\DataTypeTok{data=}\NormalTok{training2, }\DataTypeTok{method=}\StringTok{"gbm"}\NormalTok{, }\DataTypeTok{verbose=}\OtherTok{FALSE}\NormalTok{)}
\NormalTok{mod.gbm$finalModel}
\end{Highlighting}
\end{Shaded}

\begin{verbatim}
## A gradient boosted model with multinomial loss function.
## 150 iterations were performed.
## There were 40 predictors of which 36 had non-zero influence.
\end{verbatim}

\begin{Shaded}
\begin{Highlighting}[]
\NormalTok{## get predictions and confusion(accuracy matrix) for gbm}
\NormalTok{pred.gbm <-}\StringTok{ }\KeywordTok{predict}\NormalTok{(mod.gbm, testing2)}
\KeywordTok{confusionMatrix}\NormalTok{(testing2$classe, pred.gbm)}
\end{Highlighting}
\end{Shaded}

\begin{verbatim}
## Confusion Matrix and Statistics
## 
##           Reference
## Prediction    A    B    C    D    E
##          A 1373   13    5    4    0
##          B   40  873   30    2    4
##          C    0   21  814   18    2
##          D    0    8   25  769    2
##          E    3   19    3   14  862
## 
## Overall Statistics
##                                           
##                Accuracy : 0.9566          
##                  95% CI : (0.9505, 0.9621)
##     No Information Rate : 0.2887          
##     P-Value [Acc > NIR] : < 2.2e-16       
##                                           
##                   Kappa : 0.945           
##  Mcnemar's Test P-Value : 1.699e-07       
## 
## Statistics by Class:
## 
##                      Class: A Class: B Class: C Class: D Class: E
## Sensitivity            0.9696   0.9347   0.9282   0.9529   0.9908
## Specificity            0.9937   0.9809   0.9898   0.9915   0.9903
## Pos Pred Value         0.9842   0.9199   0.9520   0.9565   0.9567
## Neg Pred Value         0.9877   0.9846   0.9844   0.9907   0.9980
## Prevalence             0.2887   0.1905   0.1788   0.1646   0.1774
## Detection Rate         0.2800   0.1780   0.1660   0.1568   0.1758
## Detection Prevalence   0.2845   0.1935   0.1743   0.1639   0.1837
## Balanced Accuracy      0.9817   0.9578   0.9590   0.9722   0.9906
\end{verbatim}

\begin{Shaded}
\begin{Highlighting}[]
\NormalTok{## See what predictions look like on "real" test set}
\NormalTok{pred.gbm2 <-}\StringTok{ }\KeywordTok{predict}\NormalTok{(mod.gbm, pmltest.clean)}
\NormalTok{pred.gbm2}
\end{Highlighting}
\end{Shaded}

\begin{verbatim}
##  [1] B A B A A E D B A A B C B A E E A B B B
## Levels: A B C D E
\end{verbatim}

\begin{Shaded}
\begin{Highlighting}[]
\NormalTok{#### Ensembled Model}
\NormalTok{## Stack RPART, RF and GBM models}
\NormalTok{stacked.dat <-}\StringTok{ }\KeywordTok{data.frame}\NormalTok{(pred.rpart, pred.rf, pred.gbm, }\DataTypeTok{classe=}\NormalTok{testing2$classe)}
\KeywordTok{str}\NormalTok{(stacked.dat)}
\end{Highlighting}
\end{Shaded}

\begin{verbatim}
## 'data.frame':    4904 obs. of  4 variables:
##  $ pred.rpart: Factor w/ 5 levels "A","B","C","D",..: 1 1 1 1 1 1 1 1 1 1 ...
##  $ pred.rf   : Factor w/ 5 levels "A","B","C","D",..: 1 1 1 1 1 1 1 1 1 1 ...
##  $ pred.gbm  : Factor w/ 5 levels "A","B","C","D",..: 1 1 1 1 1 1 1 1 1 1 ...
##  $ classe    : Factor w/ 5 levels "A","B","C","D",..: 1 1 1 1 1 1 1 1 1 1 ...
\end{verbatim}

\begin{Shaded}
\begin{Highlighting}[]
\NormalTok{## Train the combined stacked predictors using random forests}
\NormalTok{mod.stack  <-}\StringTok{ }\KeywordTok{train}\NormalTok{(classe ~., }\DataTypeTok{method=}\StringTok{"rf"}\NormalTok{, }\DataTypeTok{data=}\NormalTok{stacked.dat)}

\NormalTok{## Get predictions and confusion(accuracy matrix) for stacked models}
\NormalTok{pred.stack <-}\StringTok{ }\KeywordTok{predict}\NormalTok{(mod.stack, stacked.dat)}
\KeywordTok{confusionMatrix}\NormalTok{(testing2$classe, pred.stack)}
\end{Highlighting}
\end{Shaded}

\begin{verbatim}
## Confusion Matrix and Statistics
## 
##           Reference
## Prediction    A    B    C    D    E
##          A 1392    1    2    0    0
##          B   10  936    3    0    0
##          C    0   14  838    3    0
##          D    0    0   13  791    0
##          E    0    0    0    9  892
## 
## Overall Statistics
##                                           
##                Accuracy : 0.9888          
##                  95% CI : (0.9854, 0.9915)
##     No Information Rate : 0.2859          
##     P-Value [Acc > NIR] : < 2.2e-16       
##                                           
##                   Kappa : 0.9858          
##  Mcnemar's Test P-Value : NA              
## 
## Statistics by Class:
## 
##                      Class: A Class: B Class: C Class: D Class: E
## Sensitivity            0.9929   0.9842   0.9790   0.9851   1.0000
## Specificity            0.9991   0.9967   0.9958   0.9968   0.9978
## Pos Pred Value         0.9978   0.9863   0.9801   0.9838   0.9900
## Neg Pred Value         0.9972   0.9962   0.9956   0.9971   1.0000
## Prevalence             0.2859   0.1939   0.1746   0.1637   0.1819
## Detection Rate         0.2838   0.1909   0.1709   0.1613   0.1819
## Detection Prevalence   0.2845   0.1935   0.1743   0.1639   0.1837
## Balanced Accuracy      0.9960   0.9905   0.9874   0.9909   0.9989
\end{verbatim}

\begin{Shaded}
\begin{Highlighting}[]
\NormalTok{##### Final Predictions}
\NormalTok{## Make final predictions for end of project quiz using Random Forests as it had the best out of sample accuracy}
\NormalTok{final.prediction <-}\StringTok{ }\KeywordTok{predict}\NormalTok{(mod.rf, pmltest.clean)}
\NormalTok{final.prediction}
\end{Highlighting}
\end{Shaded}

\begin{verbatim}
##  [1] B A B A A E D B A A B C B A E E A B B B
## Levels: A B C D E
\end{verbatim}

\begin{Shaded}
\begin{Highlighting}[]
\KeywordTok{summary}\NormalTok{(final.prediction)}
\end{Highlighting}
\end{Shaded}

\begin{verbatim}
## A B C D E 
## 7 8 1 1 3
\end{verbatim}

\begin{Shaded}
\begin{Highlighting}[]
\KeywordTok{Sys.time}\NormalTok{()}
\end{Highlighting}
\end{Shaded}

\begin{verbatim}
## [1] "2018-03-14 19:04:07 GMT"
\end{verbatim}

\end{document}
